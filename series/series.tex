\documentclass[a4paper, 11pt]{article}


\usepackage{amsmath}
\usepackage{amsfonts}
\usepackage[margin=2cm]{geometry}

\setlength\parindent{0pt}

\title{Series}
\author{}
\date{}

\begin{document}
\maketitle


En la antigüedad se conocían fórmulas para calcular el resultado de ciertas sumas finitas.\\

Demócrito y Arquimedes trabajaron en sumas de infinitos términos.\\


Una de las primeras series en estudiarse es la serie geométrica, en la que se suman las potencias de un número menor a la unidad ($0 < r < 1$)
$$\sum_{n \geq 0} r^n = 1 + r + r^2 + r^3 + ...$$



El número $r$ se dice que es la base de la serie geométrica.


Una motivación para estudiar series fue calcular los decimales del número pi. 

El siguiente paso fue estudiar series para aproximar funciones.

\line(1,0){400}\\

Arquímedes de Siracusa (ca. 287 - ca. 212 BC)\\
Calculó la serie geométrica con base 1/4.\\

Nicolás de Oresme (1323 - 1382)\\
Proporcionó muchos resultados sobre series.
Fue el primero en demostrar que la serie armónica era divergente.\\

Madhava de Sangamagrama (1350 - 1425)\\
Descubrió, entre otras series, las de las funciones trigonométricas seno, coseno y arcotangente.\\

François Viète (1540 - 1603)\\
Presentó el primer algoritmo infinito conocido, para calcular el valor de pi. Expuso el producto infinito\\



John Wallis (1616 - 1703)\\
Encuentra el resultado del producto infinito



James Gregory (1638 - 1675)\\
Descubrió la serie de la función arcotangente\\



Que es conocida como serie de Gregory.\\


Isaac Newton (1643 - 1717)\\
Calculó el desarrollo de un binomio con exponente fraccionario.
Aplicando su teorema, pudo expresar como serie a la función \\



Lo que le permitió cuadrar el círculo, cálculo que Wallis no había podido encontrar.\\
Con su teorema también pudo encontrar los desarrollos como serie de las funciones seno, coseno, arcoseno, etc. También para arcos de elipses, e incluso arcos y segmentos de la cuadratriz de Dinóstrato.
Newton llegó a plantear a las series como alternativa para definir lo que son las funciones.\\

Gottfried Wilhelm Leibniz (1646 - 1716)\\
Una de las primeras series que calculó fue\\



Que es la serie de Gregory para x = 1. Es llamada serie de Leibniz.\\

Brook Taylor (1685 - 1731)\\

James Stirling (1692 - 1770)\\
Publica en 1730 un trabajo sobre la llamada serie de McLaurin.\\

Leonhard Euler (1707 - 1783)\\

Joseph-Louis Lagrange (1736 - 1813)\\



Augustin-Louis Cauchy (1789 - 1857)\\



\end{document}
